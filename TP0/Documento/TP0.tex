\documentclass[12pt,a4paper]{article} %Formato de plantilla que vamos a usar

\usepackage[utf8]{inputenc}
\usepackage[spanish]{babel}
\usepackage{graphicx}
\usepackage[hidelinks]{hyperref}
\usepackage[table,xcdraw]{xcolor}
\usepackage[margin=2cm,top=2cm,bottom =2.5cm, includefoot]{geometry}
\usepackage{fancyhdr}
\usepackage{listingsutf8}

%Variables
\newcommand{\materia}{SISTEMAS OPERATIVOS Y REDES 2021}
\newcommand{\Date}{23 de Marzo de 2021}
\newcommand{\logofacu}{logo-unlp.png}
\newcommand{\unlp}{logo-universidad}
\newcommand{\ing}{logo-ing}

%Definición de colores
\definecolor{colorfacu}{HTML}{4373bf}

%Adicionales
\setlength{\headheight}{50px}
\pagestyle{fancy}
\fancyhf{}
\lhead{\includegraphics[width=4cm]{\unlp}}
\rhead{\includegraphics[width=4cm]{\ing}}
\renewcommand{\headrulewidth}{2pt}
\renewcommand{\headrule}{\hbox to\headwidth{\color{colorfacu}\leaders\hrule height \headrulewidth\hfill}}

%Formato para codigo
\definecolor{codegreen}{rgb}{0,0.6,0}
\definecolor{codegray}{rgb}{0.5,0.5,0.5}
\definecolor{codepurple}{rgb}{0.58,0,0.82}
\definecolor{backcolour}{rgb}{0.95,0.95,0.92}

\lstdefinestyle{customc}{
  belowcaptionskip=1\baselineskip,
  breaklines=true,
  frame=single,
  %xleftmargin=\parindent,
  language=C,
  showstringspaces=false,
  basicstyle=\footnotesize\ttfamily,
  keywordstyle=\bfseries\color{green!40!black},
  commentstyle=\itshape\color{purple!40!black},
  identifierstyle=\color{blue},
  stringstyle=\color{orange},
}


\begin{document}
	\cfoot{\thepage}
	\begin{titlepage}
	\centering                 
	\includegraphics[width=0.8\textwidth]{\logofacu}\par\vspace{1cm}
	{\scshape\LARGE \textbf{\materia}}\par\vspace{0.2cm}
	{\Huge\bfseries\textcolor{colorfacu}{TP n°0 G10}}\par\vspace{1cm}
	{\large Moreyra Iñigo, Juan Martín
66664/6}\par\vspace{0.2cm}
	{\large Luca Piccinini 63207/0}\par\vspace{0.2cm}
	{\large Taboada Rodney 58206/5}\par\vspace{0.2cm}
	{\large \Date\par}
	\end{titlepage}
%-------------------------------------------------------------------------------------------------	
	\clearpage
	\tableofcontents
	\clearpage
%---- Inicio de Documento-----------------%
	\section{Introducción}
	En este informe se detallara la realización de un programa que permita ingresar por línea de comando un numero IP en notación decimal, es decir un numero con el formato xxx.yyy.zzz.uuu y luego lo convierta en un número entero de 32 bits. El programa debe validar que el número ingresado sea efectivamente un número IP (los campos de 3 dígitos deben estar en el rango 0-255) y presentar el resultado en pantalla en forma hexadecimal. La presentación debe ser en la forma 0Xhhhhhhhh donde h son dígitos hexadecimales.\par
	\section{Interpretación}
	A partir del enunciado se interpreta lo siguiente:\par
	\begin{itemize}
	\item El programa debe recibir del usuario un número IP en formato “dotted-quad”, es decir, de la forma “xxx.yyy.zzz.uuu”.
    \item Cada uno de estos campos debe estar entre 0 y 255.\par
    \item La IP introducida debe ser una cadena de caracteres, ya que los puntos forman parte de ella y deben ser recibidos.
    \item El número IP se recibe por línea de comando.
    \item Errores posibles que deben ser detectados:
    	\begin{itemize}
    	\item Se ingresan letras u otros caracteres en lugar de números o puntos.
        \item Se agregan más de cuatro campos. 
        \item Se ingresa un campo mayor a 255.
        \item Se inicia o se termina con un punto.
    	\end{itemize}
        
    \item Una vez que se comprueba que no hay errores en la IP recibida, debe transformarse cada campo a binario. Cada campo tiene una longitud de 8 bits, generando en total una cadena de 32 bits, con los campos en binario concatenados. (Claro está, los puntos no se incluyen).Esto se puede lograr con la función sscanf().
    \item Finalmente, el número IP debe presentarse en formato hexadecimal. Para ello, se usa la función printf(), que hace la transformación por su cuenta.
	\end{itemize}
	\clearpage
\section{Resolución}
La resolución propuesta en pseudocódigo es la siguiente:\par
    \lstset{inputencoding=utf8/latin1, style = customc}
    \lstinputlisting[language=C]{code.c}
    Algunas aclaraciones finales:\par
    \begin{itemize}
    	\item Se usa una bandera para detectar si alguna de las verificaciones falla.\par
    	\item Nos apoyamos en función isdigit para saber si los caracteres son números del [0-9].\par
    	\item Para separar los campos (xxx, yyy, zzz, uuu) del string, y asignar sus valores a variables, se usa la función sscanf().\par
    \end{itemize}
\section{Bibliografía y más}
Para la realización de esta actividad nos apoyamos en los manuales de C y las librerías asociadas. Junto con este informe se adjunta el programa solicitado y las librerías generadas.\par 
Para la compilación de este programa usamos el compilador GCC en su version 10.2.0.\par
 A la hora de compilar recomendamos situarse en directorio raíz del programa y usar el comando, "\textit{gcc -o leerIP leeIP.c Funciones/funciones.c}", si es que se encuentra en una distribución Linux.\par
 Por último, creamos un \href{https://github.com/LucaPicc/SOyR}{\textit{repositorio en GitHub}} para los proyectos de la materia
\end{document}